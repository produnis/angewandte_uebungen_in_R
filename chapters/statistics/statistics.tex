% Author: Alfredo Sánchez Alberca (asalber@ceu.es)

\chapter{Sampling statistics}\label{cha:statistics}

\section{Solved exercises}
\begin{enumerate}[leftmargin=*]
\item The number of children in a sample of 25 families is
\begin{center}
1, 2, 4, 2, 2, 2, 3, 2, 1, 1, 0, 2, 2, 0, 2, 2, 1, 2, 2, 3, 1, 2, 2, 1, 2.
\end{center}
Do the following operations:
\begin{enumerate}
\item Create a data set with the variable \variable{children} and enter the data.
\item Compute the arithmetic mean, variance and standard deviation of the number of children.
Interpret the statistics.
\begin{indication}
\begin{enumerate}
\item Select the menu \menu{Teaching >  Descriptive statistics >  Statistics}.
\item In the dialog displayed insert the variable \variable{children} in the field \field{Variable}.
\item In the \mtab{Basic statistics} tab check the boxes of \option{Arithmetic mean}, \option{Variance} and
\option{Standard deviation}, and click the button \button{Submit}.
\end{enumerate}
\end{indication}

\item Compute the quartiles, the range, the interquartile range, the third decile and the 68th percentile. 
\begin{indication}
\begin{enumerate}
\item Select the menu \menu{Teaching >  Descriptive statistics >  Statistics}.
\item In the dialog displayed insert the variable \variable{children} in the field \field{Variable}.
\item In the \mtab{Basic statistics} tab check the boxes of \option{Quartiles}, \option{Range}, \option{Interquartile
range}, enter the values $0.3$ and $0.68$ in the field $\field{Percentiles}$, and click the button \button{Submit}.
\end{enumerate}
\end{indication}
\end{enumerate}

\item The number of people treated in the emergency service of a hospital every day of November was
\begin{center}
15 \quad 23 \quad 12 \quad 10 \quad 28 \quad 7 \quad 12 \quad 17 \quad 20 \quad 21 \quad 18 \quad 13 \quad 11 \quad 12 \quad 26 \\
30 \quad 6 \quad 16 \quad 19 \quad 22 \quad 14 \quad 17 \quad 21 \quad 28 \quad 9 \quad 16 \quad 13 \quad 11 \quad 16 \quad 20
\end{center}
Do the following operations: 
\begin{enumerate}
\item Create a data set with the variable \variable{emergencies} and enter the data.

\item Compute the arithmetic mean, variance, standard deviation and coefficient of variation of the number of
emergencies.
Interpret the statistics. 
\begin{indication}
\begin{enumerate}
\item Select the menu \menu{Teaching >  Descriptive statistics >  Statistics}.
\item In the dialog displayed insert the variable \variable{emergencies} in the field \field{Variable}.
\item In the \mtab{Basic statistics} tab check the boxes of \option{Arithmetic mean}, \option{Variance}, 
\option{Standard deviation} and \option{Coefficient of variation}, and click the button \button{Submit}.
\end{enumerate}
\end{indication}

\item Compute the coefficients of skewness and kurtosis and interpret the statistics.
\begin{indication}
\begin{enumerate}
\item Select the menu \menu{Teaching >  Descriptive statistics >  Statistics}.
\item In the dialog displayed insert the variable \variable{emergencies} in the field \field{Variable}.
\item In the \mtab{Basic statistics} tab check the boxes of \option{Coefficient of skewness} and \option{Coefficient
of kurtosis} and click the button \button{Submit}.
\end{enumerate}
\end{indication}
\end{enumerate}


\item In a group of 20 students the grades in Mathematics were
\begin{center}
SS, AP, SS, AP, AP, NT, NT, AP, SB, SS \\
SB, SS, AP, AP, NT, AP, SS, NT, SS, NT
\end{center}

Do the following operations:
\begin{enumerate}
\item  Create a data set \variable{course} with the variable \variable{grades} and enter the data.

\item  Recode the grades into scores assigning $2.5$ to SS, $6$ to AP, $8$ to NT and $9.5$ to SB.
\begin{indication}
\begin{enumerate}
\item Select the menu \menu{Teaching > Data > Variable recoding}.
\item In the dialog displayed insert the \variable{grades} in the field \field{Variable to recode}.
\item Enter the following recoding rules in the field \field{Recoding rules}:
\begin{quote}
\lstinline{"SS" = 2.5}\\
\lstinline{"AP" = 6}\\
\lstinline{"NT" = 8}\\
\lstinline{"SB" = 9.5}
\end{quote}
\item In the \field{Save new variable} click the button \button{Change}.
\item In the dialog displayed select as parent object the data set \variable{course} and click
the button \button{Accept}.
\item Enter the name \variable{score} for the new variable, uncheck the box \option{Convert in a factor} and click the
button \button{Submit}.
\end{enumerate}
\end{indication}

\item Compute the median and the interquartile range.
\begin{indication}
\begin{enumerate}
\item Select the menu \menu{Teaching >  Descriptive statistics >  Statistics}.
\item In the dialog displayed select the variable \variable{score} in the field \field{Variable}.
\item In the \mtab{Basic statistics} tab check the boxes of \option{Median} and \option{Interquartile range} and click the button \button{Submit}.
\end{enumerate}
\end{indication}
\end{enumerate}

\item The heights (in cm) of 30 students are 
\begin{center}
\begin{tabular}{ll}
Females: & 173, 158, 174, 166, 162, 177, 165, 154, 166, 182, 169, 172, 170, 168. \\
Males: & 179, 181, 172, 194, 185, 187, 198, 178, 188, 171, 175, 167, 186, 172, 176, 187.
\end{tabular}
\end{center}

Do the following operations:
\begin{enumerate}
\item Create a data set with the variables \variable{height} and \variable{gender} and enter the data.

\item Compute the arithmetic mean, median, variance, standard deviation and quartiles according to the gender.
Interpret the statistics.
\begin{indication}
\begin{enumerate}
\item Select the menu \menu{Teaching >  Descriptive statistics >  Statistics}.
\item In the dialog displayed insert the variable \variable{height} in the field \field{Variable},
check the box \option{Statistics by groups} and insert the variable \variable{gender} in the field \field{Grouping
variable(s)}.
\item In the \mtab{Basic statistics} tab check the boxes of \option{Arithmetic mean}, \option{Median},
\option{Variance}, \option{Standard deviation} and \option{Quartiles}, and click the button \button{Submit}.
\end{enumerate}
\end{indication}
\end{enumerate}

\end{enumerate}


\section{Proposed exercises}
\begin{enumerate}[leftmargin=*]
\item The number of injuries suffered by the members of a soccer team in a league were
\begin{center}
0, 1, 2, 1, 3, 0, 1, 0, 1, 2, 0, 1, 1, 1, 2, 0, 1, 3, 2, 1, 2, 1, 0, 1
\end{center}

Do the following operations:
\begin{enumerate}
\item Compute la arithmetic mean, median, variance and standard deviation of the number of injuries and interpret them.
\item Compute the coefficients of skewness and kurtosis.
\item Compute the fourth and the eighth deciles and interpret them.
\end{enumerate}

\item We want to compare the reliability of two blood pressure monitors, an arm monitor and a wrist monitor. 
For that purpose we have performed 8 repeated measures of the blood pressure of the same person with both moniors.
The measurements (in mmHg) were:
\begin{center}
\begin{tabular}{rl}
Arm monitor: & 111, 109, 112, 111, 113, 113, 114, 111\\
Wrist monitor: & 115, 113, 117, 116, 112, 112, 117, 112
\end{tabular}
\end{center}
Which monitor is more reliable?

\item The age and the marital status of a sample of 28 persons are:
\begin{center}
\begin{tabular}{|l|rrrrrrrrr|}
\hline
Marital status & \multicolumn{9}{c|}{Age}\\
\hline
Single    & 31 & 45 & 35 & 65 & 21 & 38 & 62 & 22 & 31 \\
Married     & 72 & 39 & 62 & 59 & 25 & 44 & 54 &    &    \\
Widow(er)      & 80 & 68 & 65 & 40 & 78 & 69 & 75 &    &    \\
Divorced & 31 & 65 & 59 & 58 & 50 &    &    &    &    \\
\hline
\end{tabular}
\end{center}

Do the following operations:
\begin{enumerate}
\item Compute the arithmetic mean and the standard deviation of the age according to the marital status and interpret
them.
\item What group has the most representative mean?
\end{enumerate}

\item A study wants to determine if there are relations between the blood pressure and the tobacco and drink. 
The values observed in a sample of 25 persons were:
\begin{center}
\begin{tabular}{lccccccccccccc}
\hline
Smokes  & yes & no & yes & yes & yes & no & no & yes & no & yes & no & yes & no \\
Drinks & no & no & yes & yes & no & no & yes & yes & no & yes & no & yes & yes \\
Blood pressure & 80 & 92 & 75 & 56 & 89 & 93 & 101 & 67 & 89 & 63 & 98 & 58 & 91 \\
\hline
\\
\hline
Smokes  & yes & no & no & yes & no & no & no & yes & no & yes & no & yes \\
Drink & yes & no & yes & yes & no & no & yes & yes & yes & no & yes & no \\
Blood pressure & 71 & 52 & 98 & 104 & 57 & 89 & 70 & 93 & 69 & 82 & 70 & 49 \\
\hline
\end{tabular}
\end{center}

\begin{enumerate}
\item Compute the arithmetic mean, the standard deviation and the coefficients of skewness and kurtosis of the blood
pressure for smokers and non-smokers, and interpret them.
\item Compute the same statistics for drinkers and non-drinkers. Interpret the statistics.
\item Compute the same statistics for smokers and drinkers, smokers and non-drinkers, non-smokers and drinkers, and
non-smoker and non-drinkers. Interpret the statistics
\end{enumerate}


% \item El conjunto de datos \variable{neonatos} del paquete \variable{rk.Teaching}, contiene información sobre una
% muestra de 320 recién nacidos en un hospital durante un año que cumplieron el tiempo normal de gestación. 
% Do the following operations:
% \begin{enumerate}
% \item Compute la media and la median muestral del peso de los nacidos e interpretarlos. 
% \item Compute el peso medio de los recién nacidos de la muestra según si la madre ha fumado o no durante el embarazo.
% Compute también el peso medio de los recién nacidos de madres que no han fumado durante el embarazo, según si la madre
% fumaba o no antes del embarazo. ¿Qué conclusiones se pueden sacar?
% \item ¿Cuál es la puntuación Apgar al minuto de nacer más frecuente?
% \item Compute la media de la diferencia entre las puntuaciones Apgar a los 5 minutos and al minuto de nacer. ¿Cómo
% evolucionan los recién nacidos?
% \item Compute los quartiles muestrales del peso de los recién nacidos e interpretarlos.
% \item Comparar los quartiles muestrales del peso de los recién nacidos según el gender. 
% \item ¿Por encima de qué peso estarán el 10\% de los niños con mayor peso?
% \item Si se considera que un niño es atípico por bajo peso si se encuentra entre el 5\% de los pesos más bajos, ¿por
% debajo de qué peso tiene que estar?
% \item Compute el recorrido and el range intercuartílico muestrales del peso de los recién nacidos e interpretarlos.
% \item Compute la variance and la standard deviation del peso de los recién nacidos e interpretarlos.
% \item ¿En qué grupo hay más variabilidad del peso de los recién nacidos, en las madres fumadoras o en las madres no
% fumadoras durante el embarazo? ¿En qué grupo será más representativo el peso medio?
% \item ¿Qué variable presenta más variabilidad relativa, el peso de los recién nacidos o el Apgar al minuto de nacer?
% \item Compute el coefficient of skewness and de apuntamiento muestrales del peso de los recién nacidos e interpretarlos.
% \item ¿Qué distribución es más asimétrica, la de los pesos de recién nacidos en madres mayores de 20 años o en madres
% menores de 20 años?
% \item ¿Qué distribución es más apuntada, la del peso de los recién nacidos en hombres o en mujeres?
% \item De acuerdo a la forma de la distribución, ¿puede considerarse la puntuación Apgar al minuto de nacer como una
% variable normal? ¿Y el número de cigarros fumados al día durante el embarazo?
% \end{enumerate}
\end{enumerate}

\end{enumerate}
