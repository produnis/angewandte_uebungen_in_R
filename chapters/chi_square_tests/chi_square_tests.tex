% Author: Alfredo Sánchez Alberca (asalber@ceu.es)

\chapter{Chi-square tests for comparing proportions}\label{cha:chi-square-tests}

\section{Solved exercises}
\begin{enumerate}[leftmargin=*]
% \item Given twDadas dos parejas de genes Aa and Bb, the descendencia del cruce efectuado según las leyes de Mendel, debe estar
% compuesto dthe following modo:
% \[
% \begin{tabular}{ll}
% \multicolumn{1}{c}{Fenotipo} & \multicolumn{1}{c}{Frequencies Relativas} \\
% \multicolumn{1}{c}{AB} & \multicolumn{1}{c}{9/16 = 0,5625} \\
% \multicolumn{1}{c}{Ab} & \multicolumn{1}{c}{3/16 = 0,1875} \\
% \multicolumn{1}{c}{aB} & \multicolumn{1}{c}{3/16 = 0,1875} \\
% \multicolumn{1}{c}{ab} & \multicolumn{1}{c}{1/16 = 0,0625} \\
% \end{tabular}
% \]

% Elegidos 300 individuos al azar de cierta población, se observa the siguiente distribución de frequencies:
% \[
% \begin{tabular}{ll}
% \multicolumn{1}{c}{Fenotipo} & \multicolumn{1}{c}{Frequencies Observadas} \\
% \multicolumn{1}{c}{AB} & \multicolumn{1}{c}{165} \\
% \multicolumn{1}{c}{Ab} & \multicolumn{1}{c}{47} \\
% \multicolumn{1}{c}{aB} & \multicolumn{1}{c}{67} \\
% \multicolumn{1}{c}{ab} & \multicolumn{1}{c}{21} \\
% \end{tabular}\button{
% \]

% Se pide

% \begin{enumerate}
% \item Crear un data set con las variables \variable{probability\_teorica} and \variable{frequency\_observada}.

% \item Comprobar yes esta muestra cumple las leyes de Mendel.
% \begin{indication}
% \begin{enumerate}
% \item Select the menu \texttt{Teaching > Non-parametric tests > Test Chi-square de bondad de ajuste}.
% \item In the dialog shown select the variable \variable{frequency\_observada} en el field
% \field{Frequency observada}, select the variable \variable{probability\_teorica} en el field \field{Probability
% teórica}, and hacer click en el button \button{Submit}.
% \end{enumerate}
% \end{indication}

% \item A the vista de los results del contraste, ¿se puede Submit que se cumplen las leyes de Mendel en los individuos
% de dicha población?
% \end{enumerate}


\item The table below contains the blood type of a sample of 1655 peptic ulcer patients and 10000 non-ulcer patients.
\[
\begin{tabular}{|l|l|l|l|l|}
\cline{2-5}
\multicolumn{1}{c|}{} & \multicolumn{1}{c|}{O} & \multicolumn{1}{c|}{A} & \multicolumn{1}{c|}{B} & \multicolumn{1}{c|}{AB} \\
\hline
\multicolumn{1}{|c|}{Patient} & \multicolumn{1}{c|}{911} & \multicolumn{1}{c|}{579} & \multicolumn{1}{c|}{124} & \multicolumn{1}{c|}{41} \\
\hline
\multicolumn{1}{|c|}{Controles} & \multicolumn{1}{c|}{4578} & \multicolumn{1}{c|}{4219} & \multicolumn{1}{c|}{890} & \multicolumn{1}{c|}{313} \\
\hline
\end{tabular}
\]

\begin{enumerate}
\item Create a data set with the variables \variable{blood.type} and \variable{ulcer}.

\item Perform a Chi-square hypothesis test to determine if the peptic ulcer depends on the blood type.
\begin{indication}
\begin{enumerate}
\item Select the menu \texttt{Teaching > Non-parametric tests > Chi-square test of inde\-pendence}.
\item In the dialog shown select the variable \variable{ulcer} in the field
\field{Dependent variable}, select the variable \variable{blood.type} in the field \field{Independent variable}, and click the button \button{Submit}.
\end{enumerate}
\end{indication}

\item In view of the results of the contrast, is there a relation between the peptic ulcer and the blood type?. I.e., can we affirm that the proportion of ulcer patients is different depending on the blood type?
\end{enumerate}

\item Mitchell et al. (1976, Annals of Human Biology), studied the distribution of blood types in a sample of 478 individuals of several regions in the south-west of Scotland. They got the following results:
\[
\begin{tabular}{|l|l|l|l|l|}
\cline{2-4}
\multicolumn{1}{c|}{} & \multicolumn{1}{c|}{Eskdale} & \multicolumn{1}{c|}{Annandale} & \multicolumn{1}{c|}{Nithsdale} & \multicolumn{1}{c}{} \\
\hline
\multicolumn{1}{|c|}{A} & \multicolumn{1}{c|}{33} & \multicolumn{1}{c|}{54} & \multicolumn{1}{c|}{98} & \multicolumn{1}{c|}{185} \\
\hline
\multicolumn{1}{|c|}{B} & \multicolumn{1}{c|}{6} & \multicolumn{1}{c|}{14} & \multicolumn{1}{c|}{35} & \multicolumn{1}{c|}{55} \\
\hline
\multicolumn{1}{|c|}{O} & \multicolumn{1}{c|}{56} & \multicolumn{1}{c|}{52} & \multicolumn{1}{c|}{115} & \multicolumn{1}{c|}{223} \\
\hline
\multicolumn{1}{|c|}{AB} & \multicolumn{1}{c|}{5} & \multicolumn{1}{c|}{5} & \multicolumn{1}{c|}{5} & \multicolumn{1}{c|}{15} \\
\hline
\multicolumn{1}{c|}{} & \multicolumn{1}{c|}{100} & \multicolumn{1}{c|}{125} & \multicolumn{1}{c|}{253} & \multicolumn{1}{c|}{478} \\
\cline{2-5}
\end{tabular}
\]

\begin{enumerate}
\item Create a data set with the variables \variable{blood.type} and \variable{region}.

\item Perform a Chi-square test to determine if the blood type depends on the region.
\begin{indication}
\begin{enumerate}
\item Select the menu \texttt{Teaching > Non-parametric tests > Chi-square test of inde\-pendence}.
\item In the dialog shown select the variable \variable{blood.type} in the field
\field{Dependent variable}, select the variable \variable{region} in the field \field{Independent variable}, and click the button \button{Submit}.
\end{enumerate}
\end{indication}

\item According to the outcome of the test, can we affirm that the distribution of the blood types does not depend on the region?
\end{enumerate}

\item A study tries to determine if smoking is related to the gender. 9 men and 17 women were asked. In the male sample there were 2 smokers while in the female sample there were 6 smokers. 
\begin{enumerate}
\item Create a data set with the variables \variable{gender} and \variable{smoke}.

\item Perform a Chi-square test to determine if smoking depends on the gender.  
\begin{indication}
\begin{enumerate}
\item Select the menu \texttt{Teaching > Non-parametric tests > Chi-square test of inde\-pendence}.
\item In the dialog shown select the variable \variable{smoke} in the field \field{Dependent variable}, select the variable \variable{gender} in the field \field{Independent variable}.
\item In the \mtab{Test options} tab check the box \option{Fisher's exact test}, and click the
button \button{Submit}.
\end{enumerate}
\end{indication}

\item According to the outcome of the test, is the distribution of smokers the same in both genders?

\begin{indication} In this case, the conditions to apply the Chi-square test are not satisfied because the number of smokers in the male sample is less than 5, so that we have to use the p-value of the Fisher's exact test.
\end{indication}
\end{enumerate}


\item To compare the effectiveness of two drugs against the migraine, 20 persons that suffered migraine frequently were selected, and they tried the two drugs in different moments. The table below shows the number of them that had some relief.
\begin{center}
\begin{tabular}{lcccccccccc}
\hline
Drug 1 & Yes & Yes & Yes & Yes& Yes & No & Yes & No& Yes & Yes  \\
\hline
Drug 2 & No & No & Yes & No & Yes & Yes & No & No& No & No \\
\hline\\
\hline
Drug 1 & Yes & No& Yes & No & Yes & Yes& Yes & No & Yes & Yes \\
\hline
Drug 2 & Yes & No& Yes & No & No & Yes& No & Yes & No & No\\
\hline
\end{tabular}
\end{center}

\begin{enumerate}
\item Create a data set with the variables \variable{relief.drug1}, and \variable{relief.drug2}.

\item Perform the Chi-square test to determine if the relief depends on the drug.  
\begin{indication}
\begin{enumerate}
\item Select the menu \texttt{Teaching > Non-parametric tests > Chi-square test of inde\-pendence}.
\item In the dialog shown select the variable \variable{relief.drug1} in the field \field{Dependent variable}, select the variable \variable{relief.drug2} in the field \field{Independent variable}.
\item In the \mtab{Test options} tab check the box \option{McNemar's test for paired data}, and click the button \button{Submit}.
\end{enumerate}
\end{indication}

\item According to the outcome of the test, can we affirm that the migraine relief depends on the drug? If so, which drug produces a significant higher relief?
\end{enumerate}

\end{enumerate}


\section{Proposed exercises}
\begin{enumerate}[leftmargin=*]
% \item Supongamos que queremos comprobar yes un dado está bien equilibrado o no. Lo lanzamos 1200 veces, and obtenemos the following
% results:
% \[
% \begin{tabular}{ll}
% \multicolumn{1}{c}{Número} & \multicolumn{1}{c}{Frequencies de aparición} \\
% \multicolumn{1}{c}{1} & \multicolumn{1}{c}{120} \\
% \multicolumn{1}{c}{2} & \multicolumn{1}{c}{275} \\
% \multicolumn{1}{c}{3} & \multicolumn{1}{c}{95} \\
% \multicolumn{1}{c}{4} & \multicolumn{1}{c}{310} \\
% \multicolumn{1}{c}{5} & \multicolumn{1}{c}{85} \\
% \multicolumn{1}{c}{6} & \multicolumn{1}{c}{315} \\
% \end{tabular}
% \]
% \begin{enumerate}
% \item A the vista de los results, ¿se puede Submit que el dado está bien equilibrado?
% \item Nos dicen que, en este dado, los números pares aparecen con una frequencies 3 veces superior a the de los impares. Contrastar dicha
% hipótesis.
% \end{enumerate}

\item A study tries to determine the if the presence or absence of comma when a patient arrives to the hospital influences the outcome (survive or die). The table below shows the frequencies observed in a sample:
\[
\begin{tabular}{l|l|l|l|}
\cline{2-3}
\multicolumn{1}{c|}{} & \multicolumn{1}{c|}{No comma} & \multicolumn{1}{c|}{Comma} & \multicolumn{1}{c}{} \\
\hline
\multicolumn{1}{|c|}{S} & \multicolumn{1}{c|}{484} & \multicolumn{1}{c|}{37} & \multicolumn{1}{c|}{521} \\
\hline
\multicolumn{1}{|c|}{D} & \multicolumn{1}{c|}{118} & \multicolumn{1}{c|}{89} & \multicolumn{1}{c|}{207} \\
\hline
\multicolumn{1}{c|}{} & \multicolumn{1}{c|}{602} & \multicolumn{1}{c|}{126} & \multicolumn{1}{c|}{728} \\
\cline{2-4}
\end{tabular}
\]

Is the comma on arrival at hospital a risk factor for death?

\item The recovery of a disease produced by two treatments $A$ and $B$ is classified into three categories: very good, good and bad. The treatment $A$ is applied to 32 patients and $B$ to 28. 10 out of 22 very good recoveries were with treatment $A$, 14 out of 24 good recoveries were with treatment $A$, and 8 out of 14 bad recoveries were with treatment $A$.
Is the effectiveness of the two treatments the same?

\item To determine if women are more successful than men in a subject, a sample of 10 women and 10 men has been drawn. Both have been examined by a teacher that always pass 40\% of students. Knowing that only 2 men passed, can we affirm that women are more successful than men in that subject?

\item 150 students have been asked whether they like the teaching methodology of two Biostatistics teachers. The results are shown in the table below:
\begin{center}
\begin{tabular}{|l|c|c|}
\cline{2-3}
\multicolumn{1}{c|}{Teacher 1 $\backslash$ Teacher 2} & Like & Dislike  \\
\hline
Like & 37 & 48  \\
\hline
Dislike & 44 & 21 \\
\hline
\end{tabular}
\end{center}

Can we affirm that there is a different opinion about the teachers?

\end{enumerate}

