% Author: Alfredo Sánchez Alberca (asalber@ceu.es)

\chapter{Analysis of variance (ANOVA)}\label{cha:anova}

\section{Solved exercises}
\begin {enumerate}[leftmargin=*]
\item A study tries to determine the effectiveness of three therapeutic programs $A$, $B$ and $C$ for the treatment of
acne.
The participants in the study where randomly divided into three groups and a treatment was applied to every group. 
After 16 weeks of treatment, the reduction in the percentage of acne lesions is shown in the table below.
\begin{center}
\begin{tabular}{rr|rr|rr}
\multicolumn{6}{c}{Reduction in percentage of lesions} \\
\toprule
\multicolumn{2}{c|}{Treatment $A$} & \multicolumn{2}{c|}{Treatment $B$} & \multicolumn{2}{c}{Treatment $C$} \\
48.6 & 50.8 & 68.0 & 71.9 & 67.5 & 61.4 \\
49.4 & 47.1 & 67.0 & 71.5 & 62.5 & 67.4 \\
50.1 & 52.5 & 70.1 & 69.9 & 64.2 & 65.4 \\
49.8 & 49.0 & 64.5 & 68.9 & 62.5 & 63.2 \\
50.6 & 46.7 & 68.0 & 67.8 & 63.9 & 61.2 \\
     &      & 68.3 & 68.9 & 64.8 & 60.5 \\
     &      &      &      & 62.3 &      \\
\bottomrule
\end{tabular}
\end{center}

\begin{enumerate}
\item Create a data set \variable{acne} with the variables \variable{acne.reduction} and \variable{treatment} and enter
the data of the sample.

\item Plot the means of the reduction percentage of acne lesions for every treatment.
Do you perceive differences between the treatments?
\begin{indication}
\begin{enumerate}
\item Select the menu \menu{Teaching > Charts > Means plot}.
\item In the dialog displayed insert the variable \variable{acne.reduction} in the field \field{Variable}, check the box
\option{Plot by groups} and insert the variable \variable{treatment} in the field \field{Grouping variable(s)}.
\item In the \mtab{Means options} tab uncheck the box \field{Confidence interval(s) for the mean(s)}, check the box
\option{Plot data} and click the button \button{Submit}.
\end{enumerate}
\end{indication}

\item Perform an ANOVA test to check if there are differences among the means of the reduction percentage of acne
lesions of the three treatments.
Are there significant differences among the treatments?
\begin{indication}
\begin{enumerate}
\item Select the menu \menu{Teaching > Parametric tests > Means > ANOVA}.
\item In the dialog displayed insert the data set \variable{acne} in the field \field{Data set}, check the box
\option{Between subjects} for the design, insert the variable \variable{acne.reduction} in the field \field{Dependent
variable}, the variable \variable{treatment} in the field \field{Between subjects factors} and click
the button \button{Submit}.
\end{enumerate}
There are significant differences among the treatments if the $p$-value is less than $0.05$.
\end{indication}

\item Compute the confidence intervals for the pairwise differences between means of reduction in the percentage of acne
lesions of the three treatments.
For which treatments there are significant differences?
\begin{indication}
\begin{enumerate}
\item Select the menu \menu{Teaching > Parametric tests > Means > ANOVA}.
\item In the dialog displayed insert the data set \variable{acne} in the field \field{Data set}, check the box
\option{Between subjects} for the design, insert the variable \variable{acne.reduction} in the field \field{Dependent
variable}, the variable \variable{treatment} in the field \field{Between subjects factors}.
\item In the \mtab{Pairwise comparison} tab check the boxes \option{Pairwise comparison of means} and \option{Plot
confidence intervals for the difference between the means} and click the button \button{Submit}.
\end{enumerate}
There are differences between two means if the confidence interval for the difference between the means doesn't contains
zero.
\end{indication}

\item Plot confidence intervals for the means of reduction in the percentage of acne lesions of the three
treatments. 
\begin{indication}
\begin{enumerate}
\item Select the menu \menu{Teaching > Charts > Means plot}.
\item In the dialog displayed insert the variable \variable{acne.reduction} in the field \field{Variable}, check the box
\option{Plot by groups} and insert the variable \variable{treatment} in the field \field{Grouping variable(s)}.
\item In the \mtab{Means options} tab check the box \field{Confidence interval(s) for the mean(s)} and click the button
\button{Submit}.
\end{enumerate}
\end{indication}
\end{enumerate}


\item To check if there are differences among the high schools of a city in the training for the selectivity exam, a
random sample of students of every school was drawn.
The grades in the selectivity exam of the students is shown in the following table.
\begin{center}
\begin{tabular}{ccccc}
\multicolumn{5}{c}{High schools} \\
\toprule
$A$ & $B$ & $C$ & $D$ & $E$ \\
5.5 & 6.1 & 4.9 & 3.2 & 6.7 \\
5.2 & 7.2 & 5.5 & 3.3 & 5.8 \\
5.9 & 5.5 & 6.1 & 5.5 & 5.4 \\
7.1 & 6.7 & 6.1 & 5.7 & 5.5 \\
6.2 & 7.6 & 6.2 & 6.0 & 4.9 \\
5.9 & 5.9 & 6.4 & 6.1 & 6.2 \\
5.3 & 8.1 & 6.9 & 4.7 & 6.1 \\
6.2 & 8.3 & 4.5 & 5.1 & 7.0 \\
\bottomrule
\end{tabular}
\end{center}

\begin{enumerate}
\item Create a data set \variable{selectivity} with the variables \variable{grade} and \variable{school} and enter the
data of the sample.

\item Plot the means of the selectivity exam grades for every school.
Do you perceive differences between the schools?
\begin{indication}
\begin{enumerate}
\item Select the menu \menu{Teaching > Charts > Means plot}.
\item In the dialog displayed insert the variable \variable{grade} in the field \field{Variable}, check the box
\option{Plot by groups} and insert the variable \variable{school} in the field \field{Grouping variable(s)}.
\item In the \mtab{Means options} tab uncheck the box \field{Confidence interval(s) for the mean(s)}, check the box
\option{Plot data} and click the button \button{Submit}.
\end{enumerate}
\end{indication}

\item Perform an ANOVA test to check if there are differences among the means of the selectivity exam grades of the five
high schools.
Are there significant differences among the schools?
\begin{indication}
\begin{enumerate}
\item Select the menu \menu{Teaching > Parametric tests > Means > ANOVA}.
\item In the dialog displayed insert the data set \variable{selectivity} in the field \field{Data set}, check the box
\option{Between subjects} for the design, insert the variable \variable{grade} in the field \field{Dependent
variable}, the variable \variable{school} in the field \field{Between subjects factors} and click
the button \button{Submit}.
\end{enumerate}
There are significant differences among the treatments if the $p$-value is less than $0.05$.
\end{indication}

\item Which schools are the best in the training for the selectivity exam?
\begin{indication}
\begin{enumerate}
\item Select the menu \menu{Teaching > Parametric tests > Means > ANOVA}.
\item In the dialog displayed insert the data set \variable{selectivity} in the field \field{Data set}, check the box
\option{Between subjects} for the design, insert the variable \variable{grade} in the field \field{Dependent
variable}, the variable \variable{school} in the field \field{Between subjects factors}.
\item In the \mtab{Pairwise comparison} tab check the boxes \option{Pairwise comparison of means} and \option{Plot
confidence intervals for the difference between the means} and click the button \button{Submit}.
\end{enumerate}
There are differences between two means if the confidence interval for the difference between the means doesn't contains
zero.
\end{indication}
\end{enumerate}
\end{enumerate}


\section{Proposed exercises}
\begin{enumerate}[leftmargin=*]
\item The table below shows the pulse (in beats per minute) of four groups of patients: controls ($A$), patients with 
angina pectoris ($B$), patients with heart arrhythmia ($C$) and patients recovered of a heart attack ($D$).

\begin{center}
\begin{tabular}{llll}
\toprule
$A$ & $B$ & $C$ & $D$ \\
83 & 81 & 75 & 61 \\
61 & 65 & 68 & 75 \\
80 & 77 & 80 & 78 \\
63 & 87 & 80 & 80 \\
67 & 95 & 74 & 68 \\
89 & 89 & 78 & 65 \\
71 & 103 & 69 & 68 \\
73 & 89 & 72 & 69 \\
70 & 78 & 76 & 70 \\
66 & 83 & 75 & 79 \\
57 & 91 & 69 & 61 \\
\bottomrule
\end{tabular}
\end{center}

According to the data are there significant differences among the pulse means of the four groups?


\item The table below shows the breathing frequency (breaths per minute) in a sample of lab rats exposed to three
levels of carbon monoxide.

\begin{center}
\begin{tabular}{ccc}
\multicolumn{3}{c}{Carbon monoxide level}\\
\toprule
Low & Medium & High\\
36 & 43 & 45 \\
33 & 38 & 39 \\
35 & 41 & 33 \\
39 & 34 & 39 \\
41 & 28 & 33 \\
41 & 44 & 26 \\
44 & 30 & 39 \\
45 & 31 & 29 \\
\bottomrule
\end{tabular}
\end{center}

According to the data are there significant differences among the breathing frequency means of the three levels?
\end{enumerate}
